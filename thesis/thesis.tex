\documentclass[a4paper]{article}
\usepackage[a4paper, top=1in, left=1.2in, right=1.2in, bottom=1in, footskip=0.25in]{geometry}
\usepackage[absolute]{textpos}
\usepackage{subfigure}
\usepackage{float}
\usepackage{hyperref}
\usepackage{graphicx}
\usepackage{blindtext}
\usepackage{array}
\usepackage{tabularx}
\usepackage{pgfplots}
\pgfplotsset{width=10cm,compat=1.9}
\graphicspath{ {./images/} }
\bibliographystyle{ieeetr}

\title{
	Currying the web: A custom Java REST framework - built on functional paradigms
	- compared to Spring Boot: Performance, Developer Guidance and Ease of Use
}
\author{Nico Lerchl\\2110257236\\[0.5cm]{\small Advisor: Dipl.-Ing. (FH) Bernhard Wallisch}}

\begin{document}

\begin{textblock}{15}(0.5, 0.5)
	\noindent\Large BACHELOR PAPER\\
	\large Thesis submitted in fulfillment of the requirements for the degree of Bachelor
	of Science in Engineering at the University of Applied Sciences Technikum Wien
	- Degree Program Computer Science
\end{textblock}

\maketitle

\newpage

\section*{Declaration}
“As author and creator of this work to hand, I confirm with my signature knowledge of the relevant
copyright regulations governed by higher education acts (see Urheberrechtsgesetz / Austrian
copyright law as amended as well as the Statute on Studies Act Provisions / Examination
Regulations of the UAS Technikum Wien as amended).\newline

\noindent I hereby declare that I completed the present work independently and that any ideas, whether
written by others or by myself, have been fully sourced and referenced. I am aware of any con-
sequences I may face on the part of the degree program director if there should be evidence of
missing autonomy and independence or evidence of any intent to fraudulently achieve a pass
mark for this work (see Statute on Studies Act Provisions / Examination Regulations of the UAS
Technikum Wien as amended).\newline

\noindent I further declare that up to this date I have not published the work to hand nor have I presented
it to another examination board in the same or similar form. I affirm that the version submitted
matches the version in the upload tool.“

\newpage

\section*{Kurzfassung}
\blindtext

\newpage

\section*{Abstract}
\blindtext

\newpage

\section*{Acknowledgements}
\blindtext

\newpage

\tableofcontents

\newpage

\section{Introduction}

Web development is a huge part of the software industry. Most of the time, the
server part of a web application is built using the MVC pattern and object
oriented programming \cite{damir2021architecture}. Functional programming is not
used as much in web development, in the last few years however, functional
programming has been gaining a lot of popularity with languages such as Haskell,
Scala and Clojure but also with functional concepts being added to object
oriented languages like Java and C\# \cite{klint2022functional}.\newline

\noindent Combining functional programming with web development using more
widely used languages e.g. Java is a scarcely researched topic which this thesis
aims to explore and shine some light on. The goal is to build a REST framework
in Java 21 using functional paradigms and compare it to Spring Boot in terms of
performance, developer guidance and ease of use.

\section{Literature review}
\subsection{REST}

Representational State Transfer (REST) is the state-of-the-art way to build the
server part of a client-server-architecture and it is most likely only going to
get bigger in the industry \cite{halili2018web}. It was first described by Roy
Fielding in his doctoral dissertation in 2000. REST is based on the following
properties \cite{fielding2000architectural}:

\begin{itemize}
	\item Client-server - The client and the server are separated and can be
			developed independently.
	\item Stateless - The server does not store any client state. Every request
			contains all the information the server needs to process it.
	\item Cache - Responses can be cached to improve performance.
	\item Uniform Interface - The interface between the client and the server is
			uniform and simple.
\end{itemize}

\subsection{Functional programming}
\subsubsection{General}

Functional programming - unlike procedural or object oriented programming - is
not based on the Turing machine, but rather on lambda calculus. Lambda calculus,
developed by Alonzo Church in the 1930s, is a mathematical system later proven -
by Turing himself - to be equivalent to the Turing machine
\cite{turing1937computability}.\newline

\noindent The base principles of functional programming are
\cite{hughes1989functional}:
\begin{itemize}
	\item Immutability - Variables are not changed after they are assigned a
			value.
	\item Pure functions - Functions do not have side effects and always return
			the same output for the same input.
	\item Higher order functions - Functions can be passed as arguments to other
			functions.
	\item Referential transparency - A function call can be replaced by its
			return value without changing the program's behavior.
\end{itemize}

\noindent A big part of functional programmings is the concept of monads which
have their roots in category theory. They allow for encapsulating side effects
in a pure way. For a container to be a monad it has to abide by the laws of left
identity, right identity and associativity. \cite{wadler1992essence}

\subsubsection{Web development}

Yesod is a web framework for the before mentioned functional programming
language Haskell. It allows developers to build entire websites using templates
and widgets or RESTful web services. Additionally, Yesod offers the ability to
persist data using Haskell's type system into PostgreSQL, SQLite, MySQL, and
MongoDB. \cite{snoyman2015developing}

\subsubsection{In Java}

The introduction of lambda expressions in Java 8 brought functional programming
to the Java ecosystem. Where before developers had to use anonymous classes to
pass functions as arguments, they can now use lambda expressions. This also
shifts the view point of passing an object that carries functionality to passing
behavior itself. The concept behind these lambda expressions in Java is called
functional interfaces. Functional interfaces are interfaces that have exactly
one abstract method. They can be annotated with \verb|@FunctionalInterface|.
\newline

\noindent Also new to Java 8 is the Streams API. It hides away the iteration
over collections by offering many higher order functions. Additionally, streams
are only evaluated when a terminal operation, such as collecting, counting or
averaging is called, implementing the - before mentioned - functional
programming principle of lazy evaluation. \cite{warburton2014java}

\subsection{Spring Boot}

Spring Boot is a framework for building stand-alone web applications and RESTful
web services in Java. Unlike Spring Framework there is zero requirement for XML
configuration. It can be deployed using an internal web server or going the
classic route of deploying a war file onto an external web server.
\cite{webb2013spring}

\section{Curryful}
Curryful tries to combine Java's simplicity with functional paradigms to build
greater REST APIs.

\section{Research questions and hypotheses}
\subsection{Research questions}

All of the following questions will be answered by comparing Curryful to Spring
Boot.

\begin{enumerate}
	\item Will building a REST API using functional paradigms, from the
			ground up, result in a more performant application?
	\item Will building a REST API using functional paradigms naturally
			guide the developer to eliminate unwanted behavior?
	\item Will the developer experience benefit from developing a REST API using
			functional paradigms from the ground up?
\end{enumerate}

\subsection{Hypotheses}

\begin{enumerate}
	\item Building a REST API using functional paradigms, from the ground
			up, will result in a more performant application. Functional
			programming's keenness on mutability and mitigation of side effects
			makes concurrency and parallelism disregard the need for locks or
			synchronization. In the context of FaaS, startup times are lower
			and the cold start problem is eased.
	\item Building a REST API using functional paradigms will naturally guide
			the developer to eliminate unwanted behavior. Common pitfalls of
			object oriented programming such as mutable state, side effects,
			null references and unchecked exceptions will be avoided. Null
			values are practically omitted and exceptions handled gracefully
			through the use of monads. The stateless design functional
			programming promotes will also synergize with REST's statelessness
			principle.
	\item The developer experience will benefit as error handling becomes more
			natural and testing becomes less of a burden because functions will
			always produce the same output for the same input and not rely on
			external factors. Additionally, the declarative nature of functional
			programming will increase conciseness and expressiveness directly
			leading to less lines required to achieve similar results.
\end{enumerate}

\subsection{Hypotheses}

\section{Methodology}


ChatGPT 4 was prompted to generate two simple REST APIs. One for a todo list
application and one for playing Yahtzee. Each API was generated twice, once
using Curryful and once using Spring Boot. The prompts were kept as identical as
possible besides, having to provide more information about Curryful, as ChatGPT
does not know about this new framework, resulted in changes that had to be made.

The prompts, also found in the appendix, and any changes that had to
be done to the generated code are available in the projects' repositories:
\begin{itemize}
	\item \hyperlink{https://github.com/lerchl/curryful-bachelor-thesis-curryful-todo-list}{Todo list in Curryful}
	\item \hyperlink{https://github.com/lerchl/curryful-bachelor-thesis-spring-boot-todo-list}{Todo list in Spring Boot}
	\item \hyperlink{https://github.com/lerchl/curryful-bachelor-thesis-curryful-yahtzee}{Yahtzee in Curryful}
	\item \hyperlink{https://github.com/lerchl/curryful-bachelor-thesis-spring-boot-yahtzee}{Yahtzee in Spring Boot}
\end{itemize}

\subsection{Performance}
TODO

\subsection{Provoking unwanted behavior using invalid requests}
The prompts were held short and only ask for necessary implementation details,
leaving room to play with for the AI generating the code.

Using Postman, both application generated by ChatGTP were sent requests, trying
to provoke unwanted behavior. The applications were restarted after each request
to empty the in-memory storage, which was one of the implementation details.

The requests can be found in the form of JSON, exported by Postman as a
collection v2.1
\hyperlink{https://github.com/lerchl/curryful-bachelor-thesis-postman-requests}{here}
\newline

\subsubsection{Todo list}
\noindent The requests are:
\begin{itemize}
	\item POST request where "completed" is a string
	\item POST request where a car is added
	\item POST request where the body is empty
	\item GET request for id -1
	\item PUT request for id -1
	\item POST request to toggle completed for id -1
	\item DELETE request for id -1
	\item GET request for id test
	\item GET request for id 999999999999999999999
\end{itemize}

\subsubsection{Yahtzee}

\subsection{Static code analysis}
The generated code was analyzed using SonarCloud to determine both cyclomatic
and cognitive complexity. The cyclomatic complexity is a measure of the number
of linearly independent paths through a program's source code and therefore also
represents the number of test cases required to reach a coverage of 100\%.
Cognitive complexity describes how hard it is for a person to understand the
code. (TODO: Might need citation) \newline

\noindent An additional measure to determine developer experience is the number
of lines of code and statements, which Sonar also provides. To guarantee a fair
comparison, all projects were formatted according to the same rules:

\begin{itemize}
	\item lines must not be longer than 120 characters
	\item each added part of a method chain should be in a new line, unless
	      the entire chain is not longer than 120 characters
\end{itemize}

\section{Results}
\subsection{Provoking unwanted behavior using invalid requests}
\subsubsection{Todo list}
\begin{table}[h!]
	\begin{tabularx}{\textwidth}{|X|c|c|c|}
		\hline
		\textbf{Request}                           & Expected status code & \multicolumn{2}{c|}{Actual status code}               \\
		                                           &                      & Curryful                                & Spring Boot \\
		\hline
		POST request where "completed" is a string & 400                  & 400                                     & 400         \\
		POST request where a car is added          & 400                  & 400                                     & 200         \\
		POST request where the body is empty       & 400                  & 400                                     & 400         \\
		GET request for id -1                      & 404                  & 404                                     & 200         \\
		PUT request for id -1                      & 404                  & 404                                     & 200         \\
		POST request to toggle completed for id -1 & 404                  & 404                                     & 200         \\
		DELETE request for id -1                   & 404                  & 404                                     & 200         \\
		GET request for id test                    & 400                  & -                                       & 400         \\
		GET request for id 999999999999999999999   & 400                  & -                                       & 400         \\
		\hline
	\end{tabularx}
	\caption{Results of the invalid requests}
\end{table}

\noindent The project using Curryful responded with the expected status code
seven out of nine times. The two times it did not respond with the expected
status code, the application crashed and did not respond at all. This is an
oversight by ChatGTP which tried parsing the id path parameter to an integer,
without checking if it is actually an integer:

\begin{verbatim}
	context.getPathParameters().get("id").map(Integer::parseInt)
\end{verbatim}

\noindent More than just an oversight by ChatGPT, this is a massive error in the
Curryful framework itself.
\newline

\noindent The project using Spring Boot responded with the expected status code
four out of nine times. The POST request adding a car created a todo without a
title. All requests trying to access a non-existent todo with the id -1,
returned 200, making it seem like the todo actually exists.

\subsubsection{Yahtzee}

\subsection{Static code analysis}

\section{Discussion}

\newpage

\listoffigures
\newpage

\listoftables
\newpage

\bibliography{ref}

\end{document}
